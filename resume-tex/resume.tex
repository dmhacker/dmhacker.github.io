%%%%%%%%%%%%%%%%%%%%%%%%%%%%%%%%%%%%%%%
% This is a modified ONE COLUMN version of
% the following template:
%
% Deedy - One Page Two Column Resume
% LaTeX Template
% Version 1.1 (30/4/2014)
%
% Original author:
% Debarghya Das (http://debarghyadas.com)
%
% Original repository:
% https://github.com/deedydas/Deedy-Resume
%
% IMPORTANT: THIS TEMPLATE NEEDS TO BE COMPILED WITH XeLaTeX
%
% This template uses several fonts not included with Windows/Linux by
% default. If you get compilation errors saying a font is missing, find the line
% on which the font is used and either change it to a font included with your
% operating system or comment the line out to use the default font.
%
%%%%%%%%%%%%%%%%%%%%%%%%%%%%%%%%%%%%%%
%
% TODO:
% 1. Integrate biber/bibtex for article citation under publications.
% 2. Figure out a smoother way for the document to flow onto the next page.
% 3. Add styling information for a "Projects/Hacks" section.
% 4. Add location/address information
% 5. Merge OpenFont and MacFonts as a single sty with options.
%
%%%%%%%%%%%%%%%%%%%%%%%%%%%%%%%%%%%%%%
%
% CHANGELOG:
% v1.1:
% 1. Fixed several compilation bugs with \renewcommand
% 2. Got Open-source fonts (Windows/Linux support)
% 3. Added Last Updated
% 4. Move Title styling into .sty
% 5. Commented .sty file.
%
%%%%%%%%%%%%%%%%%%%%%%%%%%%%%%%%%%%%%%%
%
% Known Issues:
% 1. Overflows onto second page if any column's contents are more than the
% vertical limit
% 2. Hacky space on the first bullet point on the second column.
%
%%%%%%%%%%%%%%%%%%%%%%%%%%%%%%%%%%%%%%
    \documentclass[]{deedy-resume-openfont}

    \begin{document}
%%%%%%%%%%%%%%%%%%%%%%%%%%%%%%%%%%%%%%
%
%     Profile
%
%%%%%%%%%%%%%%%%%%%%%%%%%%%%%%%%%%%%%%
\namesection{David}{Hacker}{dmhacker@yahoo.com | (805) 368-5071 | Thousand Oaks, CA | github.com/dmhacker}
%%%%%%%%%%%%%%%%%%%%%%%%%%%%%%%%%%%%%%
%
%     Education
%
%%%%%%%%%%%%%%%%%%%%%%%%%%%%%%%%%%%%%%
\section{Education}
\raggedright

\runsubsection{University of California, San Diego}\descript{| BS Computer Science}\hfill \location{La Jolla, CA | June 2021}\\

\sectionsep
%%%%%%%%%%%%%%%%%%%%%%%%%%%%%%%%%%%%%%
%
%     Experience
%
%%%%%%%%%%%%%%%%%%%%%%%%%%%%%%%%%%%%%%
\section{Experience}
\runsubsection{MyGolfFaves}\descript{| Application Developer}\hfill \location{Westlake Village, CA | June 2017 – Sep 2017}
\begin{tightemize}
  \item Using React Native, created iOS and Android apps for MyGolfFaves, a golfing rewards/discounts company
\end{tightemize}
\sectionsep
\runsubsection{Blinks}\descript{| Full Stack Developer}\hfill \location{Westlake Village, CA | July 2016 – Jan 2017}
\begin{tightemize}
  \item Improved the resource demands and efficiency of the Android app for Blinks, a company that used Eddystone beacons to broadcast information about retail stores and their wares
  \item When Blinks pivoted to being a subscription service for iOS stickers, I designed a functional backend to hold all of their stickers. Used the MEAN stack, mLab, Amazon S3 for image storage, and Cloudfront.
\end{tightemize}
\sectionsep
\runsubsection{IndieU}\descript{| Full Stack Developer}\hfill \location{Westlake Village, CA | Mar 2016 – Aug 2016}
\begin{tightemize}
  \item Redesigned the website for IndieU, a music sharing company. Involved numerous layout changes and required knowledge of the MEAN stack.
\end{tightemize}
\sectionsep
%%%%%%%%%%%%%%%%%%%%%%%%%%%%%%%%%%%%%%
%
%     Skills
%
%%%%%%%%%%%%%%%%%%%%%%%%%%%%%%%%%%%%%%
\section{Skills}
\raggedright
\begin{tabular}{ l l }
  \descript{Proficient in} & {\location{Java, Python, C++}}                                               \\
  \descript{Backend}       & {\location{MEAN stack, Django, Flask, Golang, Firebase}}                     \\
  \descript{Frontend}      & {\location{HTML, CSS, Javascript, Bootstrap, Material Design, React Native}} \\
\end{tabular}
\sectionsep
%%%%%%%%%%%%%%%%%%%%%%%%%%%%%%%%%%%%%%
%
%     Projects
%
%%%%%%%%%%%%%%%%%%%%%%%%%%%%%%%%%%%%%%
\section{Projects}
\raggedright

\runsubsection{\large{Photorealistic Rendering Engine}}
\descript{| Java}\hfill \location{github.com/dmhacker/RenderingEngine}\\
Ray tracer with configurable options for a variety of features: vertex normal interpolation using barycentric coordinates, Phong shading, ray reflection \& transmission, balanced k-d tree generation, camera rotation, and anti-aliasing\\
\sectionsep


\runsubsection{\large{Text Compression Experiments}}
\descript{| Python}\hfill \location{github.com/dmhacker/yatc}\\
Custom compression algorithm combining existing designs: Burrows-Wheeler transform, move-to-front transform, run-length encoding, Huffman encoding\\
\sectionsep


\runsubsection{\large{Alexa YouTube Skill}}
\descript{| Node.js, JavaScript}\hfill \location{github.com/dmhacker/alexa-youtube-skill}\\
An unpublished skill that lets Amazon Alexa devices play audio from YouTube videos\\
\sectionsep
%%%%%%%%%%%%%%%%%%%%%%%%%%%%%%%%%%%%%%
%
%     Awards
%
%%%%%%%%%%%%%%%%%%%%%%%%%%%%%%%%%%%%%%
\section{Awards}
\runsubsection{\large{1st Place @ Startup Weekend Conejo Valley 2017}} \descript{Apr 2017 | Camarillo, CA} \\
Our company, Field Vitals, used Semtech's LoRa technologies to produce concise temperature, pH, humidity, and soil moisture
readouts to maximize crop nutrients & yield for large scale farmers. We won under the Agriculture category.\\
\sectionsep
\runsubsection{\large{Honorable Mention @ SIAM M3 Challenge 2017}} \descript{Feb 2017 | New York, NY} \\
I designed a crucial regression algorithm using a Fourier series approximation, helping net our team an honorable mention (awarded to the top 8\% of teams); implemented the algorithm in Python using matplotlib and numpy; and automated data collection/processing for other parts of the challenge.\\
\sectionsep
\runsubsection{\large{3rd Place @ MIT Zero Robotics 2015}} \descript{Jan 2016 | Cambridge, MA} \\
My team went the finals hosted at MIT and ended in 3rd place out of nearly 200 international teams. Additionally, my code was run aboard the International Space Station (ISS). I designed four winning strategies for each phase in C++ and taught and directed other members of the team.\\
\sectionsep
\
\end{document}